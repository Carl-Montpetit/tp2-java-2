\documentclass[landscape]{article}
\usepackage[T1]{fontenc}
\usepackage[utf8]{inputenc}

\usepackage{tikz}

\usetikzlibrary{shapes.multipart, shapes.geometric}

\tikzset{
    edge from parent path={(\tikzparentnode.south) -| (\tikzchildnode.north)},
    every one node part/.style={font=\bfseries},
    classeNode/.style = {
        draw,
        rectangle split,
        rectangle split parts=3,
        align=left
    },
    abstract/.style = {
        font=\itshape
    },
    extends/.style = {draw, regular polygon, regular polygon sides=3}
}

\newcommand{\abstrait}[1]{\textit{#1}}

\begin{document}
\centering{
    \begin{figure}
    \begin{tikzpicture}
    \path
node[classeNode]
    {
    \nodepart{one}
\abstrait{
C}
}
[level distance=20mm]
child{
    node[extends] {}
    [level distance=20mm, sibling distance=70mm]
child{
node[classeNode]
    {
    \nodepart{one}
D\nodepart{two}
a1 : String\\
a2 : String}
[level distance=20mm]
child{
    node[extends] {}
    [level distance=20mm, sibling distance=70mm]
child{
node[classeNode]
    {
    \nodepart{one}
E\nodepart{three}
void '' m1()\\
\abstrait{
void '' m2(int, int)}
}
}
}
}
child{
node[classeNode]
    {
    \nodepart{one}
F}
}
}
    ;
    \end{tikzpicture}
    \end{figure}
}

\clearpage
\end{document}
